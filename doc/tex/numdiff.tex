\chapter{Numerical differentiation}

\section{Introduction}

Numerical differentiation refers to the algorithms for estimating the derivative of a function using only its 
values at certain evaluation points.

The simplest method for evaluating the derivative is the finite difference method. Given a function $f$
sampled at points $x_i$ ($i=0,\ldots,n-1$), its derivative is estimated using the formula

$$f'(x_i)=\frac{f(x_{i+1})-f(x_i)}{x_{i+1}-x_i}$$

For equally spaced points $x_{i+1}-x_i=h$ and the formula becomes

$$f'(x_i)=\frac{f(x_{i+1})-f(x_i)}{h}$$

There are some variations of the finite difference formula, as for example the central difference

$$f'(x_i)=\frac{f(x_{i+1})-f(x_{i-1})}{2h}$$

When solving differential equations, it is a good idea to write this expression in matrix form:

$$\left(\begin{array}{c}
f'(x_0)\\f'(x_1)\\f'(x_2)\\f'(x_3)\\\vdots\\f'(x_{n-3})\\f'(x_{n-2})\\f'(x_{n-1})
\end{array}
\right)=
\frac{1}{2h}\left(
\begin{array}{cccccccc}
-2& 2& 0& 0&\cdots&0&0&0\\
-1& 0& 1& 0&\cdots&0&0&0\\
 0&-1& 0& 1&\cdots&0&0&0\\
 0& 0&-1& 0&\cdots&0&0&0\\ 
\vdots&\vdots&\vdots&\vdots&\ddots&\vdots&\vdots&\vdots\\
 0& 0& 0& 0&\cdots&0&1&0\\ 
 0& 0& 0& 0&\cdots&-1&0&1\\
 0& 0& 0& 0&\cdots&0&-2&2
\end{array}
\right)
\left(\begin{array}{c}
f(x_0)\\f(x_1)\\f(x_2)\\f(x_3)\\\vdots\\f(x_{n-3})\\f(x_{n-2})\\f(x_{n-1})
\end{array}
\right)$$
or, in more compact form
$$f'(x_i)=\sum_{j=0}^{n-1} D_{ij} f(x_j)$$
where $D_{ij}$ is the differentiation matrix.

The finite difference method has order 2, which means that the error in the estimation of
the derivative is proportional to $h^2$. In fact, it is possible to construct higher order methods
using more points to estimate the derivative. As a rule of thumb, the order of the method
is at least equal to the number of points used in the estimation of the derivative.

Unfortunately, high order methods using equally spaced points are affected by the Runge's phenomenon. 
Indeed, finite difference formulas of order $n$ are obtained by interpolating the function between 
the points of interest using a polynomial of degree $n-1$. One of the main problems of polynomial
interpolation using polynomials of high degree is the apparition of oscillations near the edges of the 
interval between the interpolation points. The amplitude of the oscillations increase with the degree 
of the polynomial and quickly degrades the derivative estimation for high order methods.
 This effect is known as the Runge's phenomenon.

Collocation or pseudospectral methods attempt to suppress the Runge's phenomenon by choosing a set of
non-equally spaced points called collocation points. 

\subsection{Collocation/Pseudospectral methods}

Collocation methods are high order methods for estimating the derivative of a function knowing its 
values at certain points called collocation points. The position of this points is different for each
collocation method and is designed to suppress the Runge's phenomenon. A pseudospectral method
with $n$ points has order $2n$.

Each particular collocation method is associated with a family of orthogonal functions $P_l(x)$.
This functions form a basis so, any arbitrary function $\phi(x)$ can be expressed as a linear
combination of the basis functions
$$\phi(x)=\sum_{l=0}^{\infty}a_lP_l(x)$$
In practice, we will work with a finite discretization using $n$ points, so the expansion is truncated
to use only $n$ basis functions, the fuction $\phi(x)$ is then approximated by
$$\phi^{(n)}(x)=\sum_{l=0}^{n-1}\phi_lP_l(x)$$
For regular functions and a well-adapted choice of the basis functions, collocation methods have
exponential convergence, which means that the error in the approximation decreases exponentially
with the the number of basis functions $n$.

The functions $P_l(x)$ are orthogonal against some scalar product $\left<P_l,P_m\right>=\delta_{lm}$,
then the coefficients on the expansion of $\phi(x)$ can be calculated as
$$\phi_l=\left<\phi(x),P_l(x)\right>$$
For each family of basis functions it exists a formula of gaussian quadrature for calculating this
scalar product, then
$$\phi_l=\sum_{j=0}^{n-1}w_jP_l(x_j)\phi(x_j)$$
where $x_j$ and $w_j$ are the nodes and weights of the corresponding gaussian quadrature. Note
that $x_j$ are the collocation points. The estimation of the first derivative at the collocation 
points can be obtained as
$$\begin{array}{rcl}
\phi'(x_i)&=&\displaystyle\sum_{l=0}^{n-1}\phi_lP'_l(x_i)\\
&=&\displaystyle\sum_{l=0}^{n-1}\left(\sum_{j=0}^{n-1}w_jP_l(x_j)\phi(x_j)\right)P'_l(x_i)\\
&=&\displaystyle\sum_{j=0}^{n-1}\left(\sum_{l=0}^{n-1}w_jP_l(x_j)P'_l(x_i)\right)\phi(x_j)\\
&=&\displaystyle\sum_{j=0}^{n-1}D_{ij}\phi(x_j)
\end{array}$$
where $D_{ij}=\displaystyle\sum_{l=0}^{n-1}w_jP_l(x_j)P'_l(x_i)$ is the differentiation matrix.
We see that the derivative of a discretized function can be calculated
by doing a matrix product.
$$\Phi'=D\Phi$$
Similarly, the second derivative will be 
$$\Phi'=DD\Phi$$

\subsection{Relation with spectral methods}

Collocation methods are intimately related with spectral methods and share most of their properties.
The main difference is that in spectral methods we work with the coefficients $\phi_l$ in 
the expansion of the functions contrarily to collocation methods that deal directly with the values
of the function at the collocation points. This has a clear advantage when solving differential
equations with variable coefficients. For example, the equation
$$\phi'(x)+a(x)\phi(x)=b(x)$$
will be discretized using spectral methods as
$$\sum_m\left<P_l,P'_m\right>\phi_m+\sum_{m,k}\left<P_l,P_mP_k\right>a_m\phi_k=b_l$$
While the first product $\left<P_l,P'_m\right>$ use to be easy to calculate, this is not the case
for the second one $\left<P_l,P_mP_k\right>$. By contrast, for collocation methods the discretization
is just
$$\sum_jD_{ij}\phi(x_j)+a(x_i)\phi(x_i)=b(x_i)$$

It is possible to pass from one representation to the other using projection matrices. 
To get the spectral coefficients of a function $\phi$, we multiply by the projection matrix $\mathcal{P}_{ij}$.
$$\phi_l=\sum_{j=0}^{n-1}\mathcal{P}_{li}\phi(x_i)$$
where $\mathcal{P}_{li}=w_iP_l(x_i)$. To recover the values at the collocation points we do
$$\phi(x_i)=\sum_{l=0}^{n-1}\mathcal{P}^{-1}_{il}\phi_l$$
where $\mathcal{P}^{-1}_{il}=P_l(x_i)$ is the matrix inverse of $\mathcal{P}_{li}$.

\subsection{Multi-domain}

One of the main drawbacks of pseudospectral collocation methods is that they does not deal correctly
with non-regular functions. If the function that we want approximate has discontinuities, even in its
first derivatives, the exponential convergence is lost and the approximated function shows oscillations
around the discontinuity. This is known as the Gibbs phenomenon. 

There are multiple ways to deal with this problem, one of them is to use a multi-domain approach. 
It consists in dividing the integration domain in multiple subintervals. A division is placed at
the points where there is a discontinuity in the function. So now the function is continuous
in each subinterval and the pseudospectral approximation works properly.

\section{Multi-domain Gauss-Lobatto numerical differentiation}

In the Gauss-Lobatto (or more properly Gauss-Lobatto-Chebyshev) collocation method, the basis functions
are Chebyshev polynomials of the first kind
$$T_l(x)=\cos(l\arccos(x))$$
defined in the interval $(-1,1)$. The collocation points are
$$x_i=-\cos(\frac{i\pi}{n})$$
The end points of the interval are also collocation points, which make this method
well-suited for boundary value problems.

In the ESTER library, multi-domain Gauss-Lobatto numerical differentiation is implemented in the
\verb|diff_gl| class. To work with this class we should first create an object using
\mint{cpp}|diff_gl gl(n);|
The argument $n$ is optional (default 1) and indicates the number of domains. To change the number
of domains we can do
\mint{cpp}|gl.set_ndomains(n);|
After setting the number of domains we must indicate the number of points per domain and the position
of the domains. Let's see an example using three domains and the following set-up.
\begin{itemize}
\item First domain in the interval (0,5) with 30 points.
\item Second domain in the interval (5,7.5) with 20 points.
\item Third domain in the interval (7.5,10) with 20 points.
\end{itemize}
The \verb|diff_gl| object can be initialized using the code
\begin{minted}[frame=single]{cpp}
diff_gl gl;

gl.set_ndomains(3); // Use 3 domains
gl.set_xif(0.,5.,7.5,10.); // Set the limits between domains
gl.set_npts(30,20,20); // Set the number of points in each domain

gl.init(); // Initialize the object

\end{minted}

The limits between the subdomains and the number of points are stored in C arrays that are accessible 
from outside the class, so the code above is equivalent to

\begin{minted}[frame=single]{cpp}
diff_gl gl;

gl.set_ndomains(3);
gl.xif[0]=0;gl.xif[1]=5;gl.xif[2]=7.5;gl.xif[3]=10;
gl.npts[0]=30;gl.npts[1]=20;gl.npts[2]=20;

gl.init(); 

\end{minted}

During the initialization, the following objects are created:

\medskip
\begin{tabular}{cccc}
Name&Type&Size&Description\\
\hline
\texttt{ndomains}&\texttt{int}&&Number of domains\\
\texttt{N}&\texttt{int}&&Total number of points\\
\texttt{x}&\texttt{matrix}&(N,1)& Collocation points $x_i$\\
\texttt{D}&\texttt{matrix\_block\_diag}&
(N,N) in blocks of (npts[i],npts[i])&Differentiation matrix\\
\texttt{I}&\texttt{matrix}&(1,N)&Integration matrix\\
\texttt{P}&\texttt{matrix\_block\_diag}&
(N,N) in blocks of (npts[i],npts[i])&Projection matrix $\mathcal{P}_{ij}$\\
\texttt{P1}&\texttt{matrix\_block\_diag}&
(N,N) in blocks of (npts[i],npts[i])&Inverse projection matrix $\mathcal{P}^{-1}_{ij}$\\
\end{tabular}
\medskip

The following example illustrates the use of these objects:
\begin{minted}[frame=single,mathescape]{cpp}
[...] // Initialization (previous example)

matrix y,dy,ddy;

y=cos(gl.x); // Definition of function y
dy=(gl.D,y); // First derivative 
ddy=(gl.D,gl.D,y); // Second derivative 

double integral;

integral=(gl.I,y); // integral$=\int_{\mathtt{xif[0]=0}}^{\mathtt{xif[3]=10}}y(x)\mathrm{d}x$

matrix yl;

yl=(gl.P,y); // Spectral coefficients
y=(gl.P1,yl); // Recover function values from spectral coefficients

\end{minted}

A function can also be interpolated at any point within the whole domain using the method
\mint{cpp}|eval(y,x)|
where \texttt{x} can be of type \texttt{double} or \texttt{matrix} and the return value will be 
of always of type \texttt{matrix}. It returns the value of \texttt{y} at the point(s) \texttt{x}.
We can use also a third argument of type \texttt{matrix}
\mint{cpp}|eval(y,x,T)|
where \texttt{T} is modified during the call and can be used to interpolate other functions. 
For the previous example
\begin{minted}[frame=single]{cpp}
[...]

double y0,dy0;
matrix T;

y0=gl.eval(y,2.5,T)(0); // Get the value of y in x=2.5. 
						//The (0) at the end is needed because
						// the result will be a matrix of size (1,1), 
						//and we want the first (and only)
						// value of this matrix
								
dy0=(T,dy)(0); // We use the matrix T to calculate the value of dy in x=2.5

\end{minted}

A {\tt diff\_gl} object can be copied using the assignment operator. We can do
\begin{minted}{cpp}
diff_gl gl1,gl2;

gl1.set_ndomains(2);
gl1.set_npts(10,10);
gl1.set_xif(0.,0.5,1.);
gl2=gl1; 
\end{minted}
Now {\tt gl2} is an independent copy of {\tt gl1} and, as {\tt gl1} has already been initialized, this 
is also the case of {\tt gl2}.

\subsection{Example}

Let's see a full featured example that uses the class \texttt{diff\_gl}.
We are going to solve the ordinary differential equation
$$x^2\frac{\mathrm{d}^2y}{\mathrm{d}x^2}+x\frac{\mathrm{d}y}{\mathrm{d}x}-y=x^2$$
within the interval $[0,1]$ with boundary conditions
$$y(0)=0 \qquad y(1)=0$$
whose exact solution is
$$y=\frac{1}{3}x(x-1)$$
To simplify the task, we will use only 1 domain. Later, we will see the class \texttt{solver} that
allows to solve more complicated problems using several domains and several variables.

The code is the following

\begin{minted}[frame=single]{cpp}
// The following example solves the differential equation
//		x^2*y'' + x*y' - y = x^2
// with boundary conditions y(0)=0 and y(1)=0
// whose exact solution is
//      y = x*(x-1)/3		

#include<stdio.h>
#include"numdiff.h"

int main() {
	
	//Initialize a diff_gl object with 1 domain 

	int n=20; // Number of points. 
		//In fact, this example can be solved using only 3 points.
	diff_gl gl(1);
	
	gl.set_npts(n);
	gl.set_xif(0.,1.);
	gl.init();
	
	// We will work with only 1 domain, so we create a reference to the 
	// first (and only) block of gl.D
	
	matrix &D=gl.D.block(0);
	matrix &x=gl.x;
	
	// Set up the operator matrix and the right hand side
	
	matrix op,rhs;
	
	op=x*x*(D,D)+x*D-eye(n);
	rhs=x*x;
	
	// Introduce boundary conditions
	
	op.setrow(0,zeros(1,n));op(0,0)=1;
	rhs(0)=0;
	op.setrow(-1,zeros(1,n));op(-1,-1)=1;
	rhs(-1)=0;
	
	// Solve the system
	
	matrix y;
	
	y=op.solve(rhs);
	
	// Interpolate the solution into a finer grid
	
	matrix x_fine,y_fine;
	
	x_fine=vector_t(0,1,100);
	y_fine=gl.eval(y,x_fine);
	
	// Compare with the exact solution
	
	matrix y_exact;
	
	y_exact=x_fine*(x_fine-1)/3;
	
	printf("Solved using %d points\n",gl.N());
	printf("Max. error=%e\n",max(abs(y_fine-y_exact)));
	
	return 0;	
	
}
\end{minted}

To run the example, just copy the code above to a file, and compile with \texttt{ester\_build}. 
If the file is called \texttt{example.cpp} we will do
\mint{bash}|$ ester_build example.cpp -o example|     %$
and then execute using
\mint{bash}|$ ./example|     %$
The output will be something like
\begin{verbatim}
Solved using 20 points
Max. error=5.329938e-16
\end{verbatim}


\section{Gauss-Legendre numerical differentiation}

The Gauss-Legendre collocation method uses Legendre polynomials $P_l(x)$ as basis functions.
For $n$ points, the collocation points are defined as the roots of $P_n(x)$.

Legendre collocation is particularly adapted to deal with axisymmetric functions on the surface
of a sphere, that depend only on the colatitude $\theta$, just by doing $x=\cos\theta$.

The current implementation in ESTER considers only one domain, limited to one hemisphere $\theta\in[0,\pi/2]$.
This is more efficient when dealing with functions that have a defined type of symmetry 
with respect to the equator ($\theta=\pi/2$),
 which is the case for all of the variables used in the ESTER code. 
Legendre polynomials $P_l(\cos\theta)$ are symmetric with respect to $\theta=0$ (the pole). When
dealing with antisymmetric functions with respect to the pole, the derivatives 
$\frac{\mathrm{d}P_l}{\mathrm{d}\theta}$ are used as basis functions instead.

The implementation considers four types of symmetry. Each type is indicated by its own suffix.

\medskip

\begin{tabular}{cccc}
Suffix& Pole &Equator&Basis functions\\
\hline
00&Symmetric&Symmetric&$P_l(\cos\theta)$ with $l$ even \\
01&Symmetric&Antisymmetric&$P_l(\cos\theta)$ with $l$ odd\\
10&Antisymmetric&Symmetric&$\frac{\mathrm{d}P_l}{\mathrm{d}\theta}$ with $l$ odd\\
11&Antisymmetric&Antisymmetric&$\frac{\mathrm{d}P_l}{\mathrm{d}\theta}$ with $l$ even\\
\end{tabular}

\medskip

Legendre numerical differentiation is implemented in the class \texttt{diff\_leg}. In order to use
it, we must start by creating an object
\mint{cpp}|diff_leg leg;|
Then we set the number of points by setting the variable \texttt{npts}, for example
\mint{cpp}|leg.npts=20;|
and initialize the object
\mint{cpp}|leg.init();|
The following objects are created:

\medskip
\begin{tabular}{cccc}
Name&Type&Size&Description\\
\hline
\texttt{th}&\texttt{matrix}&(1,\texttt{npts})& Collocation points $x_i$\\
\texttt{D\_00, D\_01, D\_10, D\_11}&\texttt{matrix}&
(\texttt{npts},\texttt{npts})&Differentiation matrices\\
\texttt{D2\_00, D2\_01, D2\_10, D2\_11}&\texttt{matrix}&
(\texttt{npts},\texttt{npts})&Second differentiation matrices\\
\texttt{lap\_00, lap\_01, lap\_10, lap\_11}&\texttt{matrix}&
(\texttt{npts},\texttt{npts})&Laplacian operator
$\frac{1}{\sin\theta}\frac{\mathrm{d}}{\mathrm{d}\theta}\left(\sin\theta\frac{\mathrm{d}}{\mathrm{d}\theta}\right)$\\
\texttt{I\_00}&\texttt{matrix}&(\texttt{npts},1)&Integration matrix\\
\texttt{P\_00, P\_01, P\_10, P\_11}&\texttt{matrix}&
(\texttt{npts},\texttt{npts})&Projection matrices $\mathcal{P}_{ij}$\\
\texttt{P1\_00, P1\_01, P1\_10, P1\_11}&\texttt{matrix}&
(\texttt{npts},\texttt{npts})&Inverse projection matrices $\mathcal{P}^{-1}_{ij}$\\
\end{tabular}
\medskip

Contrary to the \texttt{diff\_gl} class, in which the functions was supposed to be column vectors,
the \texttt{diff\_leg} class expects functions to be defined as row vectors. This means that
all the operators are applied using right multiplication. For example, the derivative will be
\mint{cpp}|dy=(y,D_00);|
The reason for that is more clear when working with 2D functions. Consider the code
\begin{minted}[frame=single]{cpp}
int nr=30;
int nth=20;

// Inititalize a diff_gl object with nr points
diff_gl gl(1);
gl.set_xif(0.,1.);gl.set_npts(nr);
gl.init();

// Initialize a diff_leg object with nth points
diff_leg leg;
leg.npts=nth;
leg.init();

// Define a 2D function 
matrix y;
y=gl.x*sin(leg.th)*sin(leg.th);
	// gl.x is (nr,1) and leg.th is (1,nth), then y is (nr,nth)

//Compute derivatives
matrix dy_x,dy_th,dy_x_th;
dy_x=(gl.D,y); // Derivative with respect to x
dy_th=(y,leg.D_00); // Derivative with respect to th
dy_x_th=(gl.D,y,leg.D_00); // Second derivative with respect to x and th

\end{minted}

Note that the derivative of a function will have a different type of symmetry. 
For example, for a symmetric-symmetric 00 function, the first derivative is
\mint{cpp}|dy=(y,leg.D_00)|
which is of type 11. Then to calculate the second derivative we should do
\mint{cpp}|ddy=(y,leg.D_00,leg.D_11)|
or, using the second derivative matrix
\mint{cpp}|ddy=(y,leg.D2_00)|
where \texttt{ddy} has type 00.

The integration matrix is defined only for type 00 functions and computes the integral between 0 and $\pi$
with weight function $\sin\theta$

\verb|(y,leg.I_00)|$\displaystyle=\int_0^{\pi}y\sin\theta\mathrm{d}\theta$

We can interpolate functions at any point using \texttt{eval\_xx}, where \texttt{xx} is the type
of symmetry, for example
\mint{cpp}|eval_00(y,th)|
gives the value of \texttt{y} at the point(s) \texttt{th}. Here, \texttt{th} can be either 
\texttt{double} or \texttt{int} and the returned value is always of type \texttt{matrix}.
We may also use a third argument
\mint{cpp}|eval_00(y,th,T)|
where \texttt{T} can be used to interpolate additional functions at the same point(s) by doing
\mint{cpp}|(y2,T)|

A {\tt diff\_leg} object can be copied using the assignment operator. We can do
\begin{minted}{cpp}
diff_leg leg1,leg2;

leg1.npts=20;leg1.init();
leg2=leg1; 
\end{minted}
Now {\tt leg2} is an independent copy of {\tt leg1}.
\subsection{Example}

Let's see a more complete example. We will consider axisymmetric functions in spherical coordinates, 
that is functions that depend only on $r$ and $\theta$. For the radial direction we use
Gauss-Lobatto differentiation in the interval $(0,1)$, and Legendre differentiation for the
latitudinal direction. We will write two functions, one for calculating the value of the laplacian
at a certain point and another one to evaluate the volume integral within the whole sphere.
For simplicity, we will consider only type 00 functions.

The code is as follows

\begin{minted}[frame=single]{cpp}
#include<stdio.h>
#include"numdiff.h"
#include"constants.h" //For the defintion of PI

//Function prototypes
double laplacian(matrix y,double r0,double th0);
double integral(matrix y);

// Define diff_gl and diff_leg objects as global variables
diff_gl gl; 
diff_leg leg;
// Create references for spherical coordinates
matrix &r=gl.x,&th=leg.th;

int main() {

	//Initialize gl. In the example we will use 2 domains
	gl.set_ndomains(2);
	gl.set_xif(1e-12,0.2,1.); // Use 1e-12 as the interior limit (instead of 0)
							  //to avoid the central singularity
	gl.set_npts(100,100);
	gl.init();
	//Initialize leg
	leg.npts=50;
	leg.init();
	
	matrix y;
	
	//Define the function y
	y=r*r*r*(1+sin(th)*sin(th));
	
	double lap_y,int_y;
	double r0=0.3,th0=PI/3;
	
	lap_y=laplacian(y,r0,th0);
	int_y=integral(y);
	
	printf("The value of the laplacian at (%f,%f) is %e\n",r0,th0,lap_y);
	printf("The volume integral is %e\n",int_y);
	
	return 0;	
	
}

// Function for calculating the laplacian of y at (r0,th0)
double laplacian(matrix y,double r0,double th0) {

	matrix lap_y;
	
	lap_y=(gl.D,r*r*gl.D,y)/r/r+(y,leg.lap_00)/r/r;
	
	//Interpolate in the direction of r
	lap_y=gl.eval(lap_y,r0); // Now lap_y is (1,nth)
	//Interpolate in the direction of theta
	lap_y=leg.eval_00(lap_y,th0); // Now lap_y is (1,1)
	
	return lap_y(0);
	// lap_y has only 1 element, but we must include (0) 
	// at the end to return a double
}
// Function for calculating the volume integral of y
double integral(matrix y) {

	return 2*PI*(gl.I,r*r*y,leg.I_00)(0);

}

\end{minted}

After running the code, the output should be

\begin{verbatim}
The value of the laplacian at (0.300000,1.047198) is 6.150000e+00
The volume integral is 3.490659e+00
\end{verbatim}

\section{Reference}

\renewcommand{\funclistcolumns}{3}
\funclist{numdiff}

\funcrefbegin{numdiff}
\funcsec{\bf Gauss-Lobatto differentiation}
\subsection{Gauss-Lobatto differentiation}
\funcsec{\sc Data members}
\subsubsection{Data members}

\member{ndomains}{int}{Number of domains (read-only).}
\member{N}{int}{Total number of points, including all the domains (read-only).}
\member{x}{matrix}{Collocation points (N x 1).}
\member{D}{matrix\_block\_diag}{Differentiation matrix.}
\member{P}{matrix\_block\_diag}{Projection matrix.}
\member{P1}{matrix\_block\_diag}{Inverse projection matrix.}
\member{I}{matrix}{Integration matrix.}
\member{npts}{int *}{Array of size ndomains() with the number of points in each domain.}
\member{xif}{double *}{Array of size ndomains()+1 with the limits between domains.}

\funcsec{\sc Functions}
\subsubsection{Functions}

\function{set\_ndomains(n)}{Method}
{n (int): Number of domains}
{None}
{Change the number of domains.}

\function{set\_npts(n0,n1,\ldots)}{Method}
{n0,n1,\ldots (int): Number of points}
{None}
{Change the number of points in each domain.}

\function{set\_xif(x0,x1,\ldots)}{Method}
{n0,n1,\ldots (double): Number of points}
{None}
{Change the position of the limits between domains.}

\function{init()}{Method}{None}{None}{Initializes the object.}

\function{eval(y,x,T)}{Method}
{y (matrix): Function to evaluate\newline
x (matrix or double): Evaluation point(s)\newline
T (matrix): Interpolating matrix (optional, output) \newline
}{matrix}{Evaluate function y at point(s) x. If y is NxM and x is Kx1, the returned matrix is KxM.
The optional matrix T can be used to interpolate additional functions by multiplying (T,y2).} 

\funcsec{\bf Legendre differentiation}
\subsection{Legendre differentiation}
\funcsec{\sc Data members}
\subsubsection{Data members}
\member{npts}{int}{Number of points.}
\member{th}{matrix}{$\theta$ values of the collocation points (1 x npts).}
\member{P\_00,P\_01,P\_10,P\_11}{matrix}{Projection matrices for each type of symmetry.}
\member{P1\_00,P1\_01,P1\_10,P1\_11}{matrix}{Inverse projection matrices for each type of symmetry.}
\member{D\_00,D\_01,D\_10,D\_11}{matrix}{Differentiation matrices for each type of symmetry.}
\member{D2\_00,D2\_01,D2\_10,D2\_11}{matrix}{Second differentiation matrices for each type of symmetry.}
\member{lap\_00,lap\_01,lap\_10,lap\_11}{matrix}{
Laplacian operator matrices for each type of symmetry.
$$\mathtt{(lap\_xx,f)}\equiv
\frac{1}{\sin\theta}\frac{\mathrm{d}}{\mathrm{d}\theta}
\left(\sin\theta\frac{\mathrm{d}\mathtt{f}}{\mathrm{d}\theta}\right)$$
}
\member{I\_00}{matrix}{
Integration matrix (npts x 1) for symmetric functions.
$$\mathtt{(I\_00,f)}\equiv \int_0^\pi\mathtt{f}\sin(\theta)\mathrm{d}\theta$$
}
\funcsec{\sc Functions}
\subsubsection{Functions}
\function{init()}{Method}{None}{None}{Initializes the object.}
\function{eval\_00(y,th,T)}{Method}
{y (matrix): Function to evaluate\newline
th (matrix or double): Evaluation point(s)\newline
T (matrix): Interpolating matrix (optional, output) \newline
}{matrix}{Evaluate function y($\theta$) at point(s) th. 
The function y should be symmetric at the pole and the equator.
If y is MxN and x is 1xK, the returned matrix is MxK.
The optional matrix T can be used to interpolate additional functions by multiplying (T,y2).} 
\function{eval\_01(y,th,T)}{Method}
{y (matrix): Function to evaluate\newline
th (matrix or double): Evaluation point(s)\newline
T (matrix): Interpolating matrix (optional, output) \newline
}{matrix}{Evaluate function y($\theta$) at point(s) th. 
The function y should be symmetric at the pole and antisymmetric the equator.
If y is MxN and x is 1xK, the returned matrix is MxK.
The optional matrix T can be used to interpolate additional functions by multiplying (T,y2).} 
\function{eval\_10(y,th,T)}{Method}
{y (matrix): Function to evaluate\newline
th (matrix or double): Evaluation point(s)\newline
T (matrix): Interpolating matrix (optional, output) \newline
}{matrix}{Evaluate function y($\theta$) at point(s) th. 
The function y should be antisymmetric at the pole and symmetric at the equator.
If y is MxN and x is 1xK, the returned matrix is MxK.
The optional matrix T can be used to interpolate additional functions by multiplying (T,y2).} 
\function{eval\_11(y,th,T)}{Method}
{y (matrix): Function to evaluate\newline
th (matrix or double): Evaluation point(s)\newline
T (matrix): Interpolating matrix (optional, output) \newline
}{matrix}{Evaluate function y($\theta$) at point(s) th. 
The function y should be antisymmetric at the pole and the equator.
If y is MxN and x is 1xK, the returned matrix is MxK.
The optional matrix T can be used to interpolate additional functions by multiplying (T,y2).} 
\function{eval(y,th,T,par\_pol,par\_eq)}{Method}
{y (matrix): Function to evaluate\newline
th (matrix or double): Evaluation point(s)\newline
T (matrix): Interpolating matrix (output) \newline
par\_pol (int): Type of symmetry at the pole \newline
par\_eq (int): Type of symmetry at the equator
}{matrix}{Evaluate function y($\theta$) at point(s) th. 
par\_pol and par\_eq can be 0 (symmetric) or 1 (antisymmetric).
If y is MxN and x is 1xK, the returned matrix is MxK.
The matrix T can be used to interpolate additional functions by multiplying (T,y2).} 
\funcrefend

